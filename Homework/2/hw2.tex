

\documentclass{article}

\usepackage{amssymb}
\usepackage{graphicx}
\usepackage{amsmath}
\usepackage{amsfonts}
\usepackage[comma,authoryear]{natbib}
\usepackage{theorem}
\usepackage[onehalfspacing]{setspace}
\usepackage{indentfirst}
\usepackage{float}
\usepackage{geometry}
\usepackage{enumerate}
\usepackage{textcomp}


\usepackage{tikz}
\usetikzlibrary{intersections,calc}

\usepackage{mathabx}

\usepackage{url}

\setcounter{MaxMatrixCols}{10}

\newtheorem{theorem}{Theorem}
\newtheorem{acknowledgement}{Acknowledgement}
\newtheorem{algorithm}{Algorithm}
\newtheorem{axiom}{Assumption}
\newtheorem{case}{Case}
\newtheorem{claim}{Claim}
\newtheorem{conclusion}{Conclusion}
\newtheorem{condition}{Condition}
\newtheorem{conjecture}{Conjecture}
\newtheorem{corollary}{Corollary}
\newtheorem{criterion}{Criterion}
{\theorembodyfont{\rmfamily}
\newtheorem{definition}{Definition}
}
\newtheorem{example}{Example}
\newtheorem{exercise}{Exercise}
\newtheorem{lemma}{Lemma}
\newtheorem{notation}{Notation}
\newtheorem{problem}{Problem}
\newtheorem{proposition}{Proposition}
\newtheorem{remark}{Remark}
\newtheorem{solution}{Solution}
\newtheorem{summary}{Summary}
\newenvironment{proof}[1][Proof]{\noindent\textbf{#1.} }{\ \rule{0.5em}{0.5em}}
\geometry{left=1in,right=1in,top=1in,bottom=1in} 

\newcommand{\E}{\mathbb{E}}
\newcommand{\R}{\mathbb{R}}
\newcommand{\Z}{\mathbb{Z}}
\newcommand{\X}{\mathbb{X}}
\newcommand{\1}{\mathbf{1}}

\newcommand{\suchthat}{\;\ifnum\currentgrouptype=16 \middle\fi|\;}

\newcommand\invisiblesection[1]{%
  \refstepcounter{section}%
  \addcontentsline{toc}{section}{\protect\numberline{\thesection}#1}%
  \sectionmark{#1}}

\def\citeapos#1{\citeauthor{#1}'s (\citeyear{#1})}

\begin{document}

\title{Econ 210C Homework 2}
\author{Instructor: Johannes Wieland}
\date{\color{red} Due: 5/22/2024, 11:59PM PST, on Github.}
\maketitle



\section*{1. VARs}
Download data for the Federal Funds Rate, the civilian unemployment rate, and the GDP deflator inflation rate from FRED.
\begin{enumerate}[(a)]
	\item Plot the data. Make sure all graphs are appropriately labelled.
	\item Aggregate all series to a quarterly frequency by averaging over months.
\end{enumerate}
Estimate a VAR with 4 lags from 1960Q1:2007Q4. The ordering of your variables should be $\pi_t,u_t,R_t$.
\begin{enumerate}[(a)]\setcounter{enumi}{2}
	\item Briefly, explain why it would make sense to end the sample in 2007Q4?
	\item Plot the IRFs from the SVAR with the same ordering. [Optional: add 95\% error bands]
	\item Briefly, interpret your results.
	\item Plot the time series of your identified monetary shocks.
	\item What are the identified monetary shocks in 2001Q3 and 2001Q4? How should one interpret these shocks?
\end{enumerate}


\section*{2. Romer shocks}
\begin{enumerate}[(a)]
	\item Download the Romer-Romer shocks from my website and merge it with your VAR dataset. Set the values of the Romer shocks to zero before 1969Q1.
	\item Following Romer-Romer, construct the IRF from the estimation equation
	\begin{align*}
		y_t = \alpha + \sum_{s=1}^{8}\beta_s y_{t-s} + \sum_{s=0}^{12}\gamma_{s}RR_{t-s}
	\end{align*}
	where $y_t\in[\pi_t,u_t,R_t]$ are the outcome variables and $RR_t$ are the Romer shocks estimated from 1960Q1:2007Q4. [Optional: add 95\% error bands]
	\item Now estimate an SVAR ordered $RR_t,\pi_t,u_t,R_t$ with four lags from 1960Q1:2007Q4 and plot the IRFs. [Optional: add 95\% error bands]
	\item Briefly, explain why it is sensible to order the Romer shock first in the VAR.
	\item Compare the IRFs for the Romer shocks from the two methods. How are they different, and why?
	\item Compare the VAR IRFs for the Romer shocks with the VAR IRFs for the SVAR shocks in Question (1d). How are they different, and why?
	\item Compare the Romer-Romer the identified monetary shocks in 2001Q3 and 2001Q4 with the SVAR identified monetary shocks. How are they similar / different?
\end{enumerate}


\end{document}
