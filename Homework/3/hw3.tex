

\documentclass{article}

\usepackage{amssymb}
\usepackage{graphicx}
\usepackage{amsmath}
\usepackage{amsfonts}
\usepackage[comma,authoryear]{natbib}
\usepackage{theorem}
\usepackage[onehalfspacing]{setspace}
\usepackage{indentfirst}
\usepackage{float}
\usepackage{geometry}
\usepackage{enumerate}
\usepackage{textcomp}


\usepackage{tikz}
\usetikzlibrary{intersections,calc}

\usepackage{mathabx}

\usepackage{url}

\setcounter{MaxMatrixCols}{10}

\newtheorem{theorem}{Theorem}
\newtheorem{acknowledgement}{Acknowledgement}
\newtheorem{algorithm}{Algorithm}
\newtheorem{axiom}{Assumption}
\newtheorem{case}{Case}
\newtheorem{claim}{Claim}
\newtheorem{conclusion}{Conclusion}
\newtheorem{condition}{Condition}
\newtheorem{conjecture}{Conjecture}
\newtheorem{corollary}{Corollary}
\newtheorem{criterion}{Criterion}
{\theorembodyfont{\rmfamily}
\newtheorem{definition}{Definition}
}
\newtheorem{example}{Example}
\newtheorem{exercise}{Exercise}
\newtheorem{lemma}{Lemma}
\newtheorem{notation}{Notation}
\newtheorem{problem}{Problem}
\newtheorem{proposition}{Proposition}
\newtheorem{remark}{Remark}
\newtheorem{solution}{Solution}
\newtheorem{summary}{Summary}
\newenvironment{proof}[1][Proof]{\noindent\textbf{#1.} }{\ \rule{0.5em}{0.5em}}
\geometry{left=1in,right=1in,top=1in,bottom=1in} 

\newcommand{\E}{\mathbb{E}}
\newcommand{\R}{\mathbb{R}}
\newcommand{\Z}{\mathbb{Z}}
\newcommand{\X}{\mathbb{X}}
\newcommand{\1}{\mathbf{1}}

\newcommand{\suchthat}{\;\ifnum\currentgrouptype=16 \middle\fi|\;}

\newcommand\invisiblesection[1]{%
  \refstepcounter{section}%
  \addcontentsline{toc}{section}{\protect\numberline{\thesection}#1}%
  \sectionmark{#1}}

\def\citeapos#1{\citeauthor{#1}'s (\citeyear{#1})}

\begin{document}

\title{Econ 210C Homework 3}
\author{Instructor: Johannes Wieland}
\date{\color{red} Due: 05/29/2023, 11:59PM PST. Submit pdf write-up and zipped code packet on Github.}
\maketitle

\section*{1. Sticky Wage Model}
Instead of assuming that prices are sticky for one period, we now assume that nominal wages are sticky for one period,
\begin{align*}
	W_1 = W_0
\end{align*}
The short-run equilibrium is
\begin{align*}
	Y_1&=A_1N_{1}  \\
	\color{red}W_1&\color{red}= W_0  \\
	\frac{W_1}{P_1}&=A_1 \\
	Y_1&=C_1 \\
	\frac{M_1}{P_1}&=\zeta^{1/\nu}\left(1-\frac{1}{Q_1}\right)^{-1/\nu}C_{1}^{\gamma/\nu}\\
	1&=\beta E_1\left\{Q_1 \frac{P_1}{P_{2}} \frac{C_{2}^{-\gamma}}{C_{1}^{-\gamma}}\right\} 
\end{align*}
The long-run equilibrium ($t\ge 2$) is
\begin{align*}
	Y_t&=A_tN_{t}  \\
	\frac{W_t}{P_t}&= A_t  \\
	\frac{W_t}{P_t}&=\frac{\chi N_t^\varphi}{C_t^{-\gamma}} \\
	Y_t&=C_t \\
	\frac{M_t}{P_t}&=\zeta^{1/\nu}\left(1-\frac{1}{Q_t}\right)^{-1/\nu}C_{t}^{\gamma/\nu}\\
	1&=\beta E_t\left\{Q_t \frac{P_t}{P_{t+1}} \frac{C_{t+1}^{-\gamma}}{C_{t}^{-\gamma}}\right\} 
\end{align*}

\begin{enumerate}[(a)]
	\item Are firms on their labor curve? Explain.
	\item Are households on their labor supply curve? Explain.
	\item How does the labor market clear?
	\item Solve for the long-run steady state.
	\item Does the Classical Dichotomy hold in the long-run? Explain.
	\item Solve for output and the money market equilibrium in the short-run.
	\item Does the Classical Dichotomy hold in the short-run? 
	\item Explain intuitively (in words) how an increase in the money supply affects output in the short-run.
	\item How does productivity affect output? Explain intuitively.
	\item Derive the labor wedge. Is it procyclical or countercyclical?
	\item What moments of the data would you use to discriminate between the predictions of the sticky price and the sticky wage model?
\end{enumerate}



\end{document}
