\documentclass[12pt]{article}

\usepackage[utf8]{inputenc}
\usepackage{amsmath,amssymb,hyperref,array,xcolor,multicol,verbatim,mathpazo}
\usepackage[normalem]{ulem}
\usepackage[pdftex]{graphicx}
\usepackage{fullpage}

\usepackage{threeparttable}
\usepackage{geometry}
\usepackage[format=hang,font=normalsize,labelfont=bf]{caption}
\usepackage{lscape}
\usepackage{natbib}
\usepackage{setspace}
\usepackage{float,color}
\usepackage[pdftex]{graphicx}
\usepackage{pdfsync}
\usepackage{placeins}
\usepackage{geometry}
\usepackage{pdflscape}
\usepackage[normalem]{ulem}
\usepackage{threeparttable, multirow}
\useunder{\uline}{\ul}{}
\synctex=1
\usepackage{hyperref}
\hypersetup{colorlinks,linkcolor=red,urlcolor=blue,citecolor=blue}
\usepackage{bm}


\newcommand{\E}{\mathbb{E}}

% Identifying information
\title{
Homework 3
} 
\author{Linghui Wu}
\date{\today}

\begin{document}

\maketitle

\textbf{(a)} The firm is on the labor demand curve because the real wage is equal to the marginal productivity of labor 
\begin{equation}
\label{eq:labor_demand}
\frac{W_{1}}{P_{1}} = \frac{W_{0}}{P_{1}} = A_{1} = MPL_{1}.
\end{equation}

\textbf{(b)} The household is not on the labor demand curve because the labor-leisure condition does not hold in the short run when the wage is sticky
\begin{equation}
\label{eq:labor_supply}
\frac{W_{1}}{P_{1}} = \frac{W_{0}}{P_{1}} \neq \frac{\chi N^{\varphi}_{1}}{C^{-\gamma}_{1}}.
\end{equation}

\textbf{(c)} At $t=1$ with $W_{0}=W_{1}$, the firm adjusts labor demand $N^{d}$ s.t. equation (\ref{eq:labor_demand}) holds. 
The household is not on the labor supply curve and works as many hours as needed to produce the required output $Y_{1}$.

\textbf{(d)} In the long-run ($t \geq 2$), we have the following characterizations,
\begin{align*}
Y &= AN \\
\frac{W}{P} &= A \\
\frac{W}{P} &= \frac{\chi N^{\varphi}}{C^{-\gamma}} \\
Y &= C \\
\frac{M}{P} &= \zeta^{\frac{1}{\nu}}(1-\frac{1}{Q})^{-\frac{1}{\nu}}C^{\frac{\gamma}{\nu}} \\
1 &= \beta \frac{Q}{\pi}
\end{align*}
where $\pi$ denotes the steady-state inflation rate.
We can then solve for steady state,
\begin{align*}
C &= \left(\frac{A^{\varphi+1}}{\chi}\right)^{\frac{1}{\varphi+\gamma}} \\
Y &= C \\
\frac{W}{P} &= A \\
N &= \left(\frac{A^{1-\gamma}}{\chi}\right)^{\frac{1}{\varphi+\gamma}} \\
Q &= \frac{\pi}{\beta} \\
\frac{M}{P} &= \zeta^{\frac{1}{\nu}}(1-\frac{1}{Q})^{-\frac{1}{\nu}}C^{\frac{\gamma}{\nu}}
\end{align*}

\textbf{(e)} By part (d), we can see that the Classical Dichotomy holds in the long run because the real variables $Y$, $C$, $N$ and $\frac{W}{P}$ are independent of money supply $M$ and nominal interest rate $Q$.

\textbf{(f)} In the short-run, we have 
\begin{align*}
Y_{1} &= A_{1}N_{1} \\
W_{1} &= W_{0} \\
\frac{W_{1}}{P_{1}} &= A_{1} \\
Y_{1} &= C_{1} \\
\frac{M_{1}}{P_{1}} &= \zeta^{\frac{1}{\nu}}\left(1-\frac{1}{Q_{1}}\right)^{-\frac{1}{\nu}}C^{\frac{\gamma}{\nu}}_{1} \\
1 &= \beta Q_{1} \frac{P_{1}}{P} \left(\frac{C}{C_{1}}\right)^{-\gamma}
\end{align*}
where $P$ and $C$ are steady-state price levels and consumption.

Plugging $P_{1}=\frac{W_{1}}{A_{1}}$, the following system of two equations pins down $C_{1}$ and $Q_{1}$ which are required for the output and money markets clearing.
\begin{align}
\label{eq:money_demand}
\frac{M_{1}A_{1}}{W_{1}} &= \zeta^{\frac{1}{\nu}}\left(1-\frac{1}{Q_{1}}\right)^{-\frac{1}{\nu}}C^{\frac{\gamma}{\nu}}_{1} \\
\label{eq:euler_eq}
1 &= \beta Q_{1} \frac{W_{1}}{PA_{1}} \left(\frac{C}{C_{1}}\right)^{-\gamma}
\end{align}

\textbf{(g)} Part (f) suggests that the Classical Dichotomy does not hold in the short-run because the sticky wage assumption replaces the household's labor supply curve and short-term consumption is affected by nominal interest rate.

\textbf{(h)} When money supply $M_{1}$ increases, the price of money $Q_{1}$ falls. 
Equation (\ref{eq:euler_eq}) suggests that real interest rate drops, 
and the household then increases today's consumption relative to future consumption through intertemporal substitution. 
So there is an overall increase in consumption today, 
which is accommodated by the household working extra hours to produce more output.

\textbf{(i)} In the long run, an increase in productivity leads to increases in 
real wage according to part (d).

In the short-run where the wage is sticky, higher productivity results in an increase in the marginal productivity of labor, and the firm will demand less labor $N^{d}$. 
Changes in $A_{1}$ do not affect the real output and money market.  

\textbf{(j)} The labor wedge at $t=1$ is
\begin{align*}
1-\tau^{N}_{1} &= \frac{MRS_{1}}{MPL_{1}} = \chi A^{\gamma-1}_{1}N^{\gamma+\varphi}_{1}
\end{align*}
In a recession, $N_{1}$ is low and $\tau^{N}_{1}$ is high.
So the labor wedge is countercyclical in the sticky wage model.

\textbf{(k)} We can use data on unemployment and labor share of income as a proxy for marginal productivity of labor to distinguish the predictions of sticky price and sticky wage model.

\end{document}


