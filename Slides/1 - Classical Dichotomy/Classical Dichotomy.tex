\documentclass[english,xcolor=svgnames]{beamer}

\input{../../../../Templates/Latex/teachingslidesbeamer.tex}

% ===========================================================
% ===========================================================
% ===========================================================
\begin{document}

\title{Classical Dichotomy \\ Economics 210C}
\vspace{1cm}
\author[shortname]{
\begin{tabular}{c}
	Johannes Wieland \\ 
	\footnotesize \href{mailto:jfwieland@ucsd.edu}{jfwieland@ucsd.edu}  \\ 
\end{tabular}
}

\date{Spring \the\year}

\setbeamertemplate{footline}{}
\makebeamertitle
\setbeamertemplate{footline}[frame number]{}

\addtocounter{framenumber}{-1}


%%%%%%%%%%%%%%%%%%%%%%%%%%%%%%%%%%%%%%%%%%%%%%%%%%
\AtBeginSection[]{
\setbeamertemplate{footline}{}
  \frame<beamer>{ 

    \frametitle{Outline}   

    \tableofcontents[currentsection] 
  }
\setbeamertemplate{footline}[frame number]{}
\addtocounter{framenumber}{-1}
}

\AtBeginSubsection[]{
\setbeamertemplate{footline}{}
  \frame<beamer>{ 

    \frametitle{Outline}   

    \tableofcontents[currentsection,currentsubsection] 
  }
  \setbeamertemplate{footline}[frame number]{}
  \addtocounter{framenumber}{-1}
}



%\setbeamertemplate{footline}{}
%\begin{frame}
%\frametitle{Outline}   
%\tableofcontents[hideallsubsections] 
%\end{frame}
%\addtocounter{framenumber}{-1}
%\setbeamertemplate{footline}[frame number]{}

%%%%%%%%%%%%%%%%%%%%%%%%%%%%%%%%%%%%%%%%%%%%%%%%%%
\section{Introduction}
%%%%%%%%%%%%%%%%%%%%%%%%%%%%%%%%%%%%%%%%%%%%%%%%%%


\begin{frame}
\frametitle{Introduction}
\begin{itemize}
	\item So far all(?) models you have seen in the core are real.
	\item Today we will add money and prices to a standard business cycle model.
	\item The key prediction from this model is the Classical Dichotomy: money is neutral, i.e., it does not affect real variables.
	\item This result will be useful starting point for analyzing the determinants of prices and inflation (next class).
	\item We will then examine evidence for / against the Classical Dichotomy.
\end{itemize}
\end{frame}


%\begin{frame}
%\frametitle{Money}
%\begin{itemize}
%	\item Characteristics of money:
%	\begin{enumerate}
%		\item Unit of account
%		\item Medium of exchange
%		\item Store of value
%	\end{enumerate}
%	\item This class will be primarily about (1), touching occasionally on (2) and (3).
%\end{itemize}
%\end{frame}




%%%%%%%%%%%%%%%%%%%%%%%%%%%%%%%%%%%%%%%%%%%%%%%%%%
\section{Flexible Price Monetary Model}
%%%%%%%%%%%%%%%%%%%%%%%%%%%%%%%%%%%%%%%%%%%%%%%%%%

%%%%%%%%%%%%%%%%%%%%%%%%%%%%%%%%%%%%%%%%%%%%%%%%%%
\subsection{Households}
%%%%%%%%%%%%%%%%%%%%%%%%%%%%%%%%%%%%%%%%%%%%%%%%%%

\begin{frame}
\frametitle{Money Demand: Ideas}
\begin{itemize}
	\item Money has no nominal return. If bonds pay interest, why hold money?
	\begin{itemize}
		\item Money provides ``liquidity services.''
		\begin{itemize}
			\item Costly and time consuming to buy and sell bonds every time you want to buy something.
			\item You also don't want to seek out someone who wants to trade for exactly what you have.
		\end{itemize}
		\item Money provides anonymity. 
	\end{itemize}
	\item To focus on interesting questions about how money changes economy, punt on why people hold it and just put real money balances $M_t/P_t$ in the utility function.
	\begin{itemize}
		\item Choose convenient isoelastic form.
		\item Alternative: Cash in advance constraint or New Monetarist day/night markets.
	\end{itemize}
\end{itemize}
\end{frame}

\begin{frame}
\frametitle{Setup: Households}
\begin{itemize}
	\item Preferences:
	\begin{align*}
		\max_{\{C_{t+s},N_{t+s},B_{t+s},M_{t+s}\}} E_t \left\{\sum_{s=0}^{\infty}\beta^s\left(\frac{C_{t+s}^{1-\gamma}}{1-\gamma}+\zeta\frac{(M_{t+s}/P_{t+s})^{1-\nu}}{1-\nu}-\chi \frac{N_{t+s}^{1+\varphi}}{1+\varphi}\right)\right\}
	\end{align*}
	\begin{itemize}
		\item Discount factor $\beta\in(0,1)$, $\rho=-\log \beta$ is the discount rate.
		\item $\gamma>0$ is CRRA, $\sigma=1/\gamma$ is IES.
		\item $\nu>0$ determines elasticity of money demand.
		\item $\varphi>0$ where $1/\varphi$ is the Frisch elasticity of labor supply. 
	\end{itemize}
	\item Notes:
	\begin{itemize}
		\item All that matters is $U$ being twice continuously differentiable with $U_c>0,\; U_{cc}<0,\;U_m>0,\; U_{mm}<0,\; U_n<0,\; U_{nn}<0$.
	\end{itemize}
\end{itemize}
\end{frame}





\begin{frame}
\frametitle{Setup: Households}
%\begin{align*}
%		E_t \left\{\sum_{s=0}^{\infty}\beta^s\left(\frac{C_{t+s}^{1-\gamma}}{1-\gamma}+\zeta\frac{(M_{t+s}/P_{t+s})^{1-\nu}}{1-\nu}-\chi \frac{N_{t+s}^{1+\varphi}}{1+\varphi}\right)\right\}
%	\end{align*}
\begin{itemize}
	\item Budget constraint:
	\begin{align*}
		P_tC_t + B_{t}  + M_t  \le W_tN_t +  Q_{t-1}B_{t-1} + M_{t-1} + P_{t}(TR_t+PR_t)
	\end{align*}
	\item Real budget constraint:
	\begin{align*}
		C_{t}= \frac{W_t}{P_t} N_t -\frac{B_t-Q_{t-1}B_{t-1}}{P_{t}}-\frac{M_t-M_{t-1}}{P_{t}}+TR_t+PR_t
	\end{align*}
	\begin{itemize}
		\item $P_t$ is the price of output $C_t$.
		\item $B_{t}$ is holdings of a nominal bond bought at price 1 and yielding $Q_{t}$ at time $t+1$.
		\item $Q_{t}$ is gross nominal interest rate between periods $t$ and $t+1$.
		\item $W_t$ is the nominal wage. 
		\item $M_t$ is quantity of money households hold at end of period $t$.
		\item $TR_t$ are real transfers and $PR_t$ are real rebated profits.
	\end{itemize}
\end{itemize}
\end{frame}


\begin{frame}
\frametitle{A Note on Timing}
\begin{itemize}
%	\item Real budget constraint:
%	\begin{align*}
%		C_{t}= \frac{W_t}{P_t} N_t -\frac{B_t-Q_{t-1}B_{t-1}}{P_{t}}-\frac{M_t-M_{t-1}}{P_{t}}+TR_t+PR_t
%	\end{align*}
	\item Instead of giving a claim to a return on capital determined at time $t+1$, nominal bonds are coupon bonds.
\begin{itemize}
	\item Buy at face price of 1 at $t$, know that will pay $Q$ at $t+1$.
	\item Bond return between $t$ and $t+1$ now determined at time $t$.
	\item Consequently, gross $t$ to $t+1$ return is denoted as $Q_t$.
%	\item And bond holdings of bonds bought at $t$ and maturing at $t+1$ are $B_t$ not $B_{t+1}$ with the real bonds in RBC.
	\end{itemize}
	\item The (ex-post) real interest rate on the bond is not known at $t$:
	\begin{align*}
		R_{t+1}\equiv Q_t \frac{P_t}{P_{t+1}}
	\end{align*}
	\begin{itemize}
	\item Depends on the realization of inflation at $t+1$.
	\item This timing helps clarify what an ``expectation at time t'' means and is consistent with literature.
	\end{itemize}
	\item Note: Different timing from Gali.
\end{itemize}
\end{frame}


\begin{frame}
\frametitle{Household Problem: Three FOCs
}
\begin{itemize}
	\item Static FOC WRT labor:
	\begin{align*}
	\frac{W_t}{P_t}&=\frac{\chi N_t^\varphi}{C_t^{-\gamma}}
	\end{align*}
	\item Dynamic FOC WRT $B_t$: Euler equation
	\begin{align*}
		1&=\beta E_t\left\{Q_t \frac{P_t}{P_{t+1}} \frac{C_{t+1}^{-\gamma}}{C_{t}^{-\gamma}}\right\}
	\end{align*}
	\item Dynamic FOC WRT $M_t$:
	\begin{align*}
		1&=\beta E_t\left\{\frac{P_t}{P_{t+1}} \frac{C_{t+1}^{-\gamma}}{C_{t}^{-\gamma}}\right\}+\zeta\frac{(M_t/P_t)^{-\nu}}{C_{t}^{-\gamma}}
	\end{align*}
	\end{itemize}
\end{frame}


\begin{frame}
\frametitle{Intertemporal Consumption Choice}
\begin{itemize}
	\item Solve the Euler equation forward:
	\begin{align*}
		C_{t}^{-\gamma} &= E_t\left\{\beta R_{t+1}C_{t+1}^{-\gamma}\right\} \\
		&= \lim_{T\rightarrow\infty} E_t \prod_{s=0}^{T}(\beta R_{t+1+s}) C_{t+1+T}^{-\gamma}
	\end{align*}
	\item Consumption today determined by:
	\begin{itemize}
		\item Long-run consumption $C_{t+1+T}$ $\approx$ permanent income.
		\item Intertemporal substitution through the path of real interest rates.
	\end{itemize}
	\item Very different from old Keynesian consumption function, $c_t = k + mpc \times y_t$.
	\begin{itemize}
		\item Old Keynesian consumption behavior remerges in Heterogeneous Agent New Keynesian (HANK) models.
	\end{itemize}
\end{itemize}
\end{frame}


\begin{frame}
\frametitle{The Stochastic Discount Factor}
%\begin{align*}
%	\beta E_t\left\{Q_t \frac{P_t}{P_{t+1}} \frac{C_{t+1}^{-\gamma}}{C_{t}^{-\gamma}}\right\}=1
%\end{align*}
\begin{itemize}
	\item Call
	\begin{align*}
		\Lambda_{t,t+1}\equiv\frac{\beta C_{t+1}^{-\gamma}}{C_{t}^{-\gamma}} \\
	\end{align*}
	the household's \emph{stochastic discount factor}.
	
		\item Pins down economy's real interest rate as Euler is:
		\begin{align*}
		1 = \beta E_t\left\{Q_t \frac{P_t}{P_{t+1}} \frac{C_{t+1}^{-\gamma}}{C_{t}^{-\gamma}}\right\} \equiv E_t\{\Lambda_{t,t+1}R_{t+1}\}
	\end{align*}
	\begin{itemize}
		\item Very important for asset pricing, as with different assets price is determined by covariance between return and SDF.
		\item Will be implicit discount rate of firms if households own firms as this is how shareholders discount cashflows.
	\end{itemize}
\end{itemize}
\end{frame}

\begin{frame}
\frametitle{Bonds Vs. Money
}
\begin{itemize}
	\item The bonds and money FOCs can also be written as:
\begin{align*}
	U_{C_t} &= \beta Q_t E_t\left\{ \frac{P_t}{P_{t+1}} U_{C_{t+1}}\right\} \\
	U_{C_t} &= \beta E_t\left\{ \frac{P_t}{P_{t+1}} U_{C_{t+1}}\right\} + U_{M_t/P_t} 
\end{align*}
	\item Euler: MU cost of buying $\epsilon$ more bonds today = discounted price-level and return adjusted benefit of having $\epsilon$ more bonds tomorrow.
	\item Money: MU cost of holding $\epsilon$ more money today = discounted price-level adjusted benefit of having $\epsilon$ more money tomorrow plus liquidity benefits of holding $\epsilon$ more money overnight.
	\item Money Demand Trade-Off: Can buy bond and get return or money and get utility benefit
\end{itemize}
\end{frame}


\begin{frame}
\frametitle{Money Demand
}
\begin{itemize}
	\item Combining bonds and money FOCs gives money demand:
	\begin{align*}
		\frac{M_t}{P_t}=\zeta^{1/\nu}\left(1-\frac{1}{Q_t}\right)^{-1/\nu}C_{t}^{\gamma/\nu}
	\end{align*}
\begin{itemize}
	\item Increasing in $C_t$ : Consume more, demand more money.
	\item Decreasing in $Q_t$: Decreasing in opportunity cost of holding money, the nominal interest rate.
	\item Often summarize as reduced-form function:
	\begin{align*}
		\frac{M_t}{P_t}=\Phi(C_t,Q_t)
	\end{align*}
\end{itemize}
%	\item Log-linearize
%	\begin{align*}
%		\hat{m}_t-\hat{p}_t=\hat{c}_{t}^{\gamma/\nu} - \eta\hat{i}_t 
%	\end{align*}
%	where $\eta\approx \frac{1}{\nu(i+1)}$.
\end{itemize}
\end{frame}


\begin{frame}
\frametitle{Transversality Conditions
}
\begin{itemize}
	\item TVC $B_t$:
	\begin{align*}
		\lim_{T\rightarrow\infty} E_t\Lambda_{t,t+T}\frac{B_T}{P_T} = 0
	\end{align*}
	\item TVC $M_t$:
	\begin{align*}
		\lim_{T\rightarrow\infty} E_t\Lambda_{t,t+T}\frac{M_T}{P_T} = 0
	\end{align*}
	\item In words:
	\begin{itemize}
		\item Cannot borrow exponentially more and more to repay existing debt and finance consumption (rules out $<0$).
		\item Sub-optimal to save exponentially more each period (rules out $>0$).
	\end{itemize}
\end{itemize}	
\end{frame}


%%%%%%%%%%%%%%%%%%%%%%%%%%%%%%%%%%%%%%%%%%%%%%%%%%
\subsection{Firms}
%%%%%%%%%%%%%%%%%%%%%%%%%%%%%%%%%%%%%%%%%%%%%%%%%%

\begin{frame}
\frametitle{Setup: Firms}
\begin{itemize}
	\item Firms produce output $Y_t$ CRS with labor $N_t$ (no capital).
	 \begin{align*}
			Y_t = A_t N_{t} 
		\end{align*}
%	\begin{itemize}
%		\item $A_t$ is TFP, $a_t=\log A_t$ follows an AR(1) process:
%		$a_t=\rho_a a_{t-1}+\epsilon^a_t$ with $\rho_a\in[0,1]$ 
%	\end{itemize}
	\item Firms maximize profits:
	\begin{align*}
		\max_{N_t}PR_t = Y_t - \frac{W_t}{P_t}N_t 
	\end{align*}
	\item FOC:
	\begin{align*}
		 MPL_t =   A_t   = \frac{W_t}{P_t}
	\end{align*}
\end{itemize}
\end{frame}


%%%%%%%%%%%%%%%%%%%%%%%%%%%%%%%%%%%%%%%%%%%%%%%%%%
\subsection{Government}
%%%%%%%%%%%%%%%%%%%%%%%%%%%%%%%%%%%%%%%%%%%%%%%%%%

\begin{frame}
\frametitle{Setup: Government}
\begin{itemize}
	\item The government budget constraint:
	\begin{align*}
		B_t + M_t = P_t Tr_t + Q_{t-1}B_{t-1} + M_{t-1}
	\end{align*}
	\item In real terms:
	\begin{align*}
		\frac{B_t}{P_t} + \frac{M_t}{P_t} =  Tr_t + \frac{Q_{t-1}B_{t-1}}{P_t} + \frac{M_{t-1}}{P_t}
	\end{align*}
	\item Government issues bonds or prints money to finance transfers and pay off past debt.
	\item Taxes are captured as negative transfers.
%	\item Household TVCs rule out ponzi schemes.
%\begin{align*}
%	\frac{B_t}{P_t}  + \sum_{s=0}^{\infty}\frac{Tr_{t+s}}{\prod_{j=0}^s R_{t+1+j}} = \lim_{T\rightarrow\infty}\frac{B_T / P_T}{\prod_{j=0}^T R_{t+1+j}} \underbrace{=}_{TVC} 0
%\end{align*}
\end{itemize}
\end{frame}


%%%%%%%%%%%%%%%%%%%%%%%%%%%%%%%%%%%%%%%%%%%%%%%%%%
\subsection{Markets}
%%%%%%%%%%%%%%%%%%%%%%%%%%%%%%%%%%%%%%%%%%%%%%%%%%

\begin{frame}
\frametitle{Setup: Markets}
\begin{itemize}
	\item Four markets
	\begin{itemize}
		\item Labor: $N_t^{firms}=N_{t}^{households}$
		\item Bond: $B_t^{government}=B_{t}^{households}$
		\item Money: $M_t^{government}=M_t^{households}$
		\item Output: $Y_t=C_t$.
	\end{itemize}
\end{itemize}
\end{frame}

%%%%%%%%%%%%%%%%%%%%%%%%%%%%%%%%%%%%%%%%%%%%%%%%%%
\subsection{Equilibrium}
%%%%%%%%%%%%%%%%%%%%%%%%%%%%%%%%%%%%%%%%%%%%%%%%%%

\begin{frame}
\frametitle{Equilibrium Definition}
An equilibrium is an allocation $\{C_{t+s},N_{t+s},Y_{t+s},B_{t+s}\}_{s=0}^{\infty}$, a set of prices $\{W_{t+s},P_{t+s},Q_{t+s}\}_{s=0}^{\infty}$, an exogenous processes $\{A_{t+s},Tr_{t+s},M_{t+s}\}_{s=0}^{\infty}$ and initial conditions for bonds $B_{t-1}$ such that:
\begin{enumerate}[1.]
	\item Households maximize utility subject to budget constraints.
	\item Firms maximize profits given their technology.
	\item The government satisfies its budget constraint.
	\item Markets clear:
	\begin{enumerate}[3.1.]
		\item Labor demanded equals labor supplied.
		\item Bond issuance by the government equals bond holding by households.
		\item Money issuance by the government equals money holdings by households.
		\item Output equals consumption plus investment.
	\end{enumerate}
\end{enumerate}
\end{frame}


\begin{frame}
\frametitle{Equilibrium Equations}
\begin{align*}
	Y_t&=A_tN_{t}  \\
	\frac{W_t}{P_t}&= A_t  \\
	\frac{W_t}{P_t}&=\frac{\chi N_t^\varphi}{C_t^{-\gamma}} \\
	Y_t&=C_t \\
	\frac{M_t}{P_t}&=\zeta^{1/\nu}\left(1-\frac{1}{Q_t}\right)^{-1/\nu}C_{t}^{\gamma/\nu}\\
	1&=\beta E_t\left\{Q_t \frac{P_t}{P_{t+1}} \frac{C_{t+1}^{-\gamma}}{C_{t}^{-\gamma}}\right\} 
\end{align*}
\end{frame}

%%%%%%%%%%%%%%%%%%%%%%%%%%%%%%%%%%%%%%%%%%%%%%%%%%
\section{Classical Dichotomy}
%%%%%%%%%%%%%%%%%%%%%%%%%%%%%%%%%%%%%%%%%%%%%%%%%%

\begin{frame}
\frametitle{Classical Dichotomy: The Real Block
}
\begin{itemize}
	\item First four equations in the ``real block'' pin down output, employment, the real wage, and consumption.
	\begin{align*}
		\text{Labor Supply: }\frac{W_t}{P_t}&=\frac{\chi N_t^\varphi}{C_t^{-\gamma}}\\
		\text{Labor Demand: }\frac{W_t}{P_t}&= A_tN_{t} 
	\end{align*}
	and $C_t=Y_t=A_tN_t $
	\item Yields one equation in $N_t$:
	\begin{align*}
		\frac{\chi N_t^\varphi}{(A_tN_t )^-\gamma}= A_t \;\Rightarrow\;N_t=\left(\frac{1}{\chi}A_{t}^{1-\gamma}\right)^{\frac{1}{\varphi+\gamma }}
	\end{align*}
	\item Monetary Neutrality: Real outcomes are independent of the price level and unaffected by nominal variables.
%	\begin{itemize}
%		\item Super Neutrality: Not only does the level of money not matter, but its long-run growth rate also does not matter.
%	\end{itemize}
	\item Monetary Neutrality implies the Classical Dichotomy: can analyze real and nominal variables independently.
\end{itemize}
\end{frame}




\begin{frame}
\frametitle{How General Is The Neutrality Result?
}
\begin{itemize}
	\item Look at real block imposing $Y_t=C_t$ :
	\begin{align*}
	C_t&=F(N_t;A_t) \\
	\frac{W_t}{P_t}&=F_N(N_t;A_t) \\
	\frac{W_t}{P_t}&=\frac{U_{N_t}}{U_{C_t}} 
\end{align*}
	\item Money cannot show up in aggregate resource constraint or
production function.
	\item From labor supply, see key condition is \emph{separability between
money and consumption / labor in utility function.}
	\begin{itemize}
		\item $M_t/P_t$ does not affect $U_{N_t}$ or $U_{C_t}$ and thus does not affect MRS or labor supply curve.
	\end{itemize}
\end{itemize}
\end{frame}


\begin{frame}
\frametitle{Non-Separable Money In Utility
}
\begin{itemize}
	\item We can generate non-neutrality from creating a cross-partial between $M_t/P_t$ and $N_t$ or $C_t$ in the utility function.
	\begin{itemize}
		\item See Gali book.
%		\item Relationship between money supply and price level violates empirical evidence.
	\end{itemize}
	\item I find this to be a fairly unsatisfying way to obtain
non-neutrality.
	\begin{itemize}
		\item Money in the utility function is a short cut.
		\item What does it mean for liquidity services to increase when labor or consumption changes?
		\item I used to think that the liquidity services effect was unimportant, but recent work has convinced me that it is important in emerging economies.
		\begin{itemize}
			\item Chodorow-Reich, Gopinath, Mishra, Narayanan (2019) in India
			\item Alvarez and Argente (2019) in Mexico
		\end{itemize}	
	\end{itemize}
\end{itemize}
\end{frame}



\begin{frame}
\frametitle{What About Capital?
}
\begin{itemize}
	\item Assuming away capital is not driving neutrality.
	\item Non-capital equations and resource constraint:
	\begin{align*}
	Y_t&=F(K_{t-1},N_t;A_t) \\
	\frac{U_{N_t}}{U_{C_t}}&=F_N(K_{t-1},N_t;A_t) \\
	Y_t&=C_{t}+K_t-(1-\delta)K_{t-1} 
	\end{align*}
	\item Household Euler is in terms of real rate, as is firm capital FOC:
	\begin{align*}
		1&=E_t\left\{\frac{\beta C_{t+1}^{-\gamma}}{C_{t}^{-\gamma}}R_{t+1}\right\} \\
		R_{t+1}&=F_K(K_{t},N_{t+1};A_{t+1})+(1-\delta)
	\end{align*}
	\item Real side pinned down independent of nominal side.
\end{itemize}
\end{frame}


%%%%%%%%%%%%%%%%%%%%%%%%%%%%%%%%%%%%%%%%%%%%%%%%%%
\section{Next Steps}
%%%%%%%%%%%%%%%%%%%%%%%%%%%%%%%%%%%%%%%%%%%%%%%%%%


\begin{frame}
\frametitle{Next Steps
}
\begin{itemize}
	\item For now we embrace the Classical Dichotomy and will learn how to solve the model, and analyze how prices and inflation are determined in our economy.
	\begin{itemize}
		\item Classical Dichotomy makes our life simpler here, without meaningfully changing conclusions.
	\end{itemize}
	\item Then we will at evidence for / against monetary neutrality.
\end{itemize}
\end{frame}

\end{document}